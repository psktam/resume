\documentclass[11pt]{article}
\usepackage[a4paper, total={7.5in, 10.5in}]{geometry}
\usepackage{hyperref}
\usepackage[T1]{fontenc}
\usepackage[compact]{titlesec}
\renewcommand{\familydefault}{cmss}

\begin{document}
\thispagestyle{empty}
\noindent
{\Huge \textbf{Peter Kim}} \hfill \textbf{peter.s.kim427@gmail.com}
\vspace{3pt}
\hrule

\section*{Education}
B.S. Mechanical Engineering, 2014; University of California - Berkeley

\section*{Experience}
    \subsubsection*{Starship QD Automation
                    (October 2021 - Present) \hfill SpaceX}
        \begin{itemize}
            \setlength\itemsep{-0.5em}
            \item Write control software for hydraulic systems that
                  connect launch vehicle to ground-side commodities.
            \item Develop and implement forward and inverse kinematics 
                  models for control of 3 DOF mechanism for second 
                  stage.
            \item Create and maintain fluid system models used to estimate
                  system performance.
            \item Write integrated hardware-out-of-the-loop tests in 
                  Python.
            
        \end{itemize}
        
    \subsubsection*{Octagrabber Autocoupler Controls 
                    (January 2020 - October 2021) \hfill SpaceX}
        \begin{itemize}
            \setlength\itemsep{-0.5em}
            \item Designed and implemented control system for mechanism that
                  remotely couples ground-side fluids to 
                  booster.
            \item System challenges: bulkheads involve dozens of 
                  connections that have very narrow alignment constraints.
                  Required ability to maintain or correct alignment
                  even in highly dynamic sea states.
            \item Developed code that allows robot to lift and
                  reposition rocket as needed for alignment.
            \item System comprises 8 servo motors, 8 pneumatic actuators,
                  10 laser sensors, 6 load cells, 10 cameras.
            \item Software written in TwinCAT and deployed on Beckhoff PLC. 
                  Schematics drawn in EPlan ProPanel.
        \end{itemize}

    \subsubsection*{Octagrabber Controls Commissioning
                    (March 2020 - October 2021) \hfill SpaceX}
		\begin{itemize}
		    \setlength\itemsep{-0.5em}
		    \item Commissioned control system for rocket-securing robot
		          currently deployed on SpaceX droneships.
            \item Integrated system that comprises over 20 servo motors, 
                  50 sensors, hydraulic track drive system, and
                  8 cameras.
            \item System deployed using Beckhoff/TwinCAT/EtherCAT technology
                  stack.
		    \item Managed part procurement; project closed out in time to 
		          support recovery mission even with COVID chip shortage 
		          issues.
		    
		\end{itemize}
	
	\subsubsection*{Octagrabber Operations Development 
	                (January 2019 - October 2021) \hfill SpaceX}
        \begin{itemize}
            \setlength\itemsep{-0.5em}
            \item Served as primary octagrabber operator in order to 
                  optimize securing operation. 
            \item Optimized booster securing process from 10 hours with two operators to 2.5 hours with single operator.
            \item Developed software in C++ to process LiDAR data to localize 
                  rocket relative to the robot.
            \item Authored training program now used to train new robot
                  operators.
            \item Personally secured over 20 boosters successfully,
                  even in highly unstable conditions.
        \end{itemize}

    \subsubsection*{Launch Data Review Automation
                    (October 2014 - October 2019) \hfill SpaceX}
        \begin{itemize}
            \setlength\itemsep{-0.5em}
            \item Implemented time-series analysis algorithms for hydraulics
                  systems at launch sites.
            \item Processed data collected from 3 launch sites with hundreds
                  of channels each in order to facilitate critical go/no-go
                  analyses for launch.
            \item Developed web-frontend to allow engineers to interact with
                  corpus of historical data. Currently used across the company
                  and also by commercial and government customers.
            \item Implemented in python. Notable packages are
                  numpy, scipy, scikit-learn, and matplotlib.
            \item Developed novel algorithm for robust time-series alignment 
                  using Taylor expansions.
        \end{itemize}

\subsubsection*{3D Printing \hfill Personal Hobby}
    \begin{itemize}
        \setlength\itemsep{-0.5em}
        \item  Designed and built custom 3D printers.
               \url{https://github.com/psktam/3d_printer_mk3}
        \item Current build: RAMPS1.4 running Marlin firmware, heated bed,
              FDM printer with all-metal hotend
        \item Part files designed in SolidWorks and fabricated with laser
              cutter.
    \end{itemize}

\section*{Skill Summary}
    \begin{itemize}
        \setlength\itemsep{-0.5em}
        \item \textbf{Python}: time-series signal processing via numpy, scipy,      
              matplotlib
        \item \textbf{Fluid Systems Analysis and Simulation}: hydraulics, pneumatics, 
              cryogenic propellants
        \item \textbf{Embedded Systems}: PLCs, Arduinos, C++/C
        \item \textbf{Fabrication}: Tube and sheet-metal bending, 
              3D printing, laser-cutting, water-jetting, CNC milling.
        \item \textbf{CAD}: SolidWorks, AutoDesk Inventor, Siemens NX
    \end{itemize}


\end{document}
